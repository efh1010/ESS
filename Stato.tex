%configurazioni	
	\documentclass[10pt,a4paper]{report}
	\usepackage{graphicx}
	\usepackage[utf8]{inputenc}
	\usepackage{amssymb,amsmath}
	\usepackage{geometry}
	\usepackage[italian]{babel}  
	\usepackage[bookmarks=true]{hyperref}
	\usepackage{bookmark}
	\geometry{letterpaper}	

\hypersetup{
	pdftitle={Elettronica dello stato solido},%
	pdfauthor={Edoardo Contini},%
	pdfsubject={Elettronica dello stato solido},%
	pdfkeywords={},%
	colorlinks=true,%
	linkcolor=blue,%
	linktocpage=true,%
	pageanchor=true
}


\begin{document}


\begin{titlepage}

	\centering
	{\scshape\huge\textbf Politecnico di Milano \par}
	\vspace{0.5cm}

	\includegraphics[width=0.15\textwidth]{logo.png}\par\vspace{0.2cm}
	
	{\scshape\small Facoltà di ingegneria\par}
	{\scshape\small Dipartimento di elettronica e informazione\par}
	\vspace{1.5cm}
	{\huge\bfseries Elettronica dello stato solido\par}
	\vspace{1.5cm}
	{\scshape Appunti\par}
	{\scshape\small 2015-2016 \par}
	\vspace{2cm}
	%{\Large\itshape Edoardo Contini\par}
	\vfill
    \raggedleft{Professore: \\ \textit{Daniele Ielimini}}	
    \\[1cm]
	
	\raggedright
    {Studente:\\ \textit{Edoardo Contini}
	
	}\vfill
	\raggedleft
	{\large \today\par}
	
	\raggedright
	\textit{La risposta è dentro di voi (ma è sbagliata)}.
	{\\Corrado Guzzanti}
	\end{titlepage}

%\maketitle


\tableofcontents

\pdfbookmark{\contentsname}{Indice}


\chapter{Strutture Cristalline}

	\section{Reticoli di Bravais}
	
		\subsection{Basi}
			Avendo un reticolo che non è un reticolo di Bravais, è possibile in alcuni casi definire una base che faccia diventare il reticolo una struttura di Bravais.
		
		\subsection{Cella unitaria}

			Si definisce cella primitiva unitaria di un reticolo un volume di spazio che, traslato attraverso tutti i vettori di un reticolo di Bravais, riempie completamente il reticolo senza sovrapposizioni e senza lasciare spazi vuoti.

		\subsection{Cella primitiva unitaria}
			Una cella primitiva è una cella unitaria che contiene un solo punto del reticolo, ed ha stessa simmetria del reticolo.
		
		\subsection{Cella di Wigner-Seitz}
				La cella di Wigner-Seitz attorno ad un punto di un reticolo di Bravais è la cella primitiva che gode di tutte le proprietà di simmetria della struttura.
			
			
			%	\centering
			%	{\includegraphics[width=0.15\textwidth]{Wigner.png}}
			
	
	\section{Reticoli notevoli}
		
		\subsection{Reticolo cubico semplice}
			

		\subsection{Reticolo cubico a facce centrate}
	
			%\begin{figure}
			%	\centering
			%	{\includegraphics[width=0.15\textwidth]{fcc.png}}
			%\end{figure}
	
		\subsection{Reticolo cubico a corpo centrato}

	
		\subsection{Struttura Diamante}
				La struttura del diamante è formata da due FCC compenetrati lungo la diagonale del cubo di un quarto della diagonale stessa.
			
		\subsection{Struttura zincomblenda}
				La struttura dello zincoblenda è simile a quella del diamante con la differenza che mentre nel diamante la base è formata da sua atomi identici nello zincoblenda la base è formata da due atomi diversi.

				Questo tipo di struttura è tipica dei semiconduttori formati da elementi del terzo del quinto gruppo.

		\section{Densità atomica}
		La densità atomica si calcola come:

		\[
		n=\frac{\#_{atomi \ per \ cella}}{volume \ cella}
		\]

		\subsection{Reticolo esagonale compatto}

	%		\begin{figure}
	%			\centering
	%			{\includegraphics[width=0.15\textwidth]{hcc.png} }
	%		\end{figure}


\chapter{Reticolo reciproco}

\chapter{Diffrazione di Bragg}

	\section{Angoli di diffrazione}
	Dalla condizione di interferenza distruttiva si ottengono gli angoli associati ai picchi di diffrazione..

	\[
	2dsin(\theta)=n\lambda
	\]


	\section{Distanza tra piani reticolo cubico semplice}

	Per quanto riguarda il reticolo cubico semplice è facile calcolare la distanza tra piani con la sequente:

	\[
	d=\frac{1}{\sqrt{a^2+b^2+c^3}}
	\]

	Dove "a" "b" e "c" sono gli indici di miller che indentificano le famiglie di piani.
\chapter{Relazione di De Broglie}

\[
p=\frac {h}{\lambda}
\]


\chapter{Effetto fotoelettrico}

\chapter{Buche di potenziale}

	\section{Approssimazione buca a pareti infinite}

	\section{Buche a pareti finite}

	Per considerare la limitatezza delle pareti è necessario eseguire un algoritmo iterativo sulle 2 equazioni seguenti:

	\[
		E_n=\frac{h^2}{\sqrt{8m(a+\Delta x)^2}}
	\]
	In cui $\Delta x$ è la penetrazione dell'autofunzione nella barriera calcolata come:
	\[
	\Delta x = \frac{1}{\alpha}=\frac{\hslash}{\sqrt{2m(V-E)}}
	\]

	\section{Buca parabolica}
	
	\[
	E_n = \sqrt{\frac{\alpha}{m}}\hslash(n+\frac{1}{2})
	\]

\chapter{Buche accoppiate}

	\section{Soluzione pari}
		\[
		tan(ka)=0
		\]
	\section{Soluzioni dispari}
		\[
		tan(ka)= - \frac {\hslash ^ 2 k} {m u}
		\]
	\section{Splitting di energia}
		\[
		E_{n_{pari}}=E_{n_{dispari}} - \Delta E
		\]
	\section{Frequenza di oscillazione}
		\[
		\nu=\frac{\Delta E}{h} 
		\]



\chapter{Probabilità di tunneling}

	\section{Barriera triangolare}

	\begin{equation}
	P_{tun}=e^{-\frac{4}{3} \frac { \sqrt{2m} } {\hslash q F} \Phi^ { \frac {3} {2} } }
	\end{equation}
	
\chapter{Flussi e coefficienti di riflessione}

	\section{Incidenza su un gradino di tensione positivo}

		\subsection{Flusso incidente}

		\subsection{Flusso trasmesso}

		\subsection{Flusso riflesso}

	
	\section{Incidenza su un gradino di tensione negativo}


		\subsection{Flusso incidente}

		\subsection{Flusso trasmesso}

		\subsection{Flusso riflesso}

	\section{Incidenza su una barriera di tensione}

		\subsection{Flusso incidente}

		\subsection{Flusso trasmesso}

		\subsection{Flusso riflesso}


\chapter{Principio di indeterminazione di Heisenberg}

\chapter{Transizioni di banda}

 	\subsection{Aprossimazione a buca parabolica}

 	\[
 	E=\frac{\hslash^2 k^2}{2 m^*}
 	\]

\chapter{Minimi di conduzione}

	\section{Massa conduzione}

	Calcolo della massa DOS in presenza di minimi non isotropi.

	\section{Massa DOS}
	\begin{equation}
	M_{DOS}^{\frac{3}{2}}=gm_t m_l^{\frac{1}{2}}
	\end{equation}

	Calcolo della massa DOS in prensenza di bande multiple.

	\begin{equation}
	M_{DOS}^{\frac{3}{2}} = m_{hh}^{\frac{3}{2}}+m_{lh}^{\frac{3}{2}}
	\end{equation}
	
	\section{Energia media}

\chapter{Relazione di Drude}

\[
\frac{\textit{F}}{\hslash}=\frac{dk}{dt}+\frac{k}{\tau_m}
\]

\chapter{Iniezione di portatori in eccesso}

	\section{Iniezione costante nel tempo}

	\section{Fotogeneraione deltiforme}
	\[
	p'(x,t)=\frac{N_g e^{-\frac {t}{\tau_p} } }{\sqrt{4\pi D_p t}}e^{-\frac{x^2+\mu_p F t}{4 D_p t}}
	\]
	\section{Inizione superficiale}

	\section{Iniezione uniforme}

\chapter{Regime di freeze-out}



\chapter{Regime intrinseco}

\chapter{Funzioni di Bloch}

u_k

	\section{Ampiezza oscillazioni di Bloch}

	\section{Pulsazione oscillazioni di Bloch}

\chapter{Effetto Hall}

Ricordiamo che nell'esperimento di Hall entrambi i portatori vengono spinti dalla forza di Lorentz nella stessa direzione.
Posso quindi determinare la tipologia di drogaggio a secondo del segno della tensione di Hall rilevata.

\chapter{Variazioni di Temperatuta}

	\section{Mobilità}

		\subsection{Scattering fononico}



		\subsection{Urti con atmoni ionizzati}



	\section{Densità di stati equivalente}

	\section{Concentrazione di portatori}


\chapter{Distribuzione di Fermi-Dirac}

	Conoscendo la probablità di occupazione di un determiato stato di energia è possibile determinare la temperatura del sistema:

	\[
	P(occupazione)= \frac {1} { 1 + e^ { \frac{E-E_F} {KT} }}
	\]

\chapter{Livelli donori}

\chapter{Livello di Fermi}

\chapter{Livello di quasi-Fermi}

\chapter{Probabilità di ionizzazione}

 	\section{Concentrazione di atomi ionizzati}

 	\[
 	N_{D_0} = \frac {N_D} { 1+ \frac {1} {2} e^ {-\frac {E_D-E_F} {KT} } } = 2N_De^{-\frac{E_D-E_F}{KT}}
 	\]

 	\section{Concentrazione di atomi non ionizzati}

 	\[
 	N_D^+= N_De^{-\frac{E_C-E_F}{KT}}
 	\]

 	\section{Probabilità}

 	\[
 	P_{ni}=\frac{N_{D_0}}{N_{D_0}+N_D^+}
 	\]

\chapter{Metalli}

	\section{Livello di Fermi}

		\subsection{Monodimensionale}

		\subsection{Bidimensionale}

		\subsection{Tridimensionale}

\chapter{Potenza dissipata per scattering fononico}

 	\section{Numero di fononi}

 	Conoscendo l'energia del singolo fonone $\hslash \omega$:



\chapter{Relazione di dispersione}

	\section{Velocità di fase}

	\section{Velocità di gruppo}

	\section{Massa efficace}


\chapter{Atomo idrogenoide}
	
	\section{Atomo di idrogeno}

	L'ultimo stato dell'atomo di idrogeno\footnote{costante di Riedberg} è a energia $13,6eV$.
	Possiamo quindi calcolare gli stati in un atomo di idrogeno come :

	\[
	E_n=-\frac{13.6eV}{n^2}
	\] 

	\section{Atomo idrogenoide}
	Per il calcolo di energia e raggio si parte da tre equazioni:

	\[
	E=-R_y\frac{\frac{m^*}{m_0}}{\epsilon_r}
	\]

	Conosciamo inoltre una relazione che lega il raggio e l'energia del rispettivo stati eccitato:

	\[
	E=-\frac{q^2}{8\pi \epsilon}\frac{1}{r}
	\]

	Inoltre si puo ricavare da partire dalla quantizzazione di Bohr e l'imposizione di orbita circolare:



	\[
	R=\frac{4\pi \epsilon_0 \epsilon_r \hslash^2 }{q^2 m}
	\]

	\[
	E=-\frac{q^4 m}{2\hslash^2 (4\pi\epsilon_o\epsilon_r)^2}
	\]

	\begin{itemize}
	\item Quantizzazione di Bohr:   $mrv=\hslash$
	\item Orbita circolare\footnote{Uguaglianza tra forza centripeta e forza elettrostatica}:	$ m \frac {v^2} {R} = \frac{q^2}{4\pi \epsilon r^2}$
	\item Energia totale:	$E=E_c+E_p=\frac{1}{2}mv^2+qv=\frac{1}{2}mv^2-\frac{q^2}{4\pi \epsilon r}$
	\end{itemize}
		
		\subsection{Raggio}

		\[
		R=\frac{4\pi \epsilon \hslash^2 }{q^2 m}	
		\]

		\subsection{Energia}

		\[
		E=\frac{mq^4}{2(4\pi \epsilon)^2 \hslash^2}
		\]

\chapter{Temperatura e velocità di Fermi}

\chapter{Velocità e campo di Saturazione}

\chapter{Eccitoni}



\chapter{Pacchetti d'onda}









\end{document}